% \iffalse meta-comment
%% unifront.dtx
%% Copyright 2023 Weifan Wei
%
% This work may be distributed and/or modified under the
% conditions of the LaTeX Project Public License, either version 1.3
% of this license or (at your option) any later version.
% The latest version of this license is in
%   https://www.latex-project.org/lppl.txt
% and version 1.3c or later is part of all distributions of LaTeX
% version 2008 or later.
%
% This work has the LPPL maintenance status `maintained'.
% 
% The Current Maintainer of this work is Weifan Wei
%
% This work consists of the files unifront.dtx and unifront.sty.
% \fi
% \iffalse
%<*driver>
\ProvidesFile{unifront.dtx}[2023-05-07 v1.0 Unique front structure]
\documentclass{ltxdoc}
\title{An Unique Front Structure}
\author{Weifan Wei}
\date{2023/05/07}
\begin{document}
\maketitle
  \DocInput{unifront.dtx}
\end{document}
%</driver>
% \fi
% \section{Introduction}
% This package can help you format your notes easily and give them a unique front page for every chapter and section.
%It automatically calculates your progress and then displays the progress bar under the title.You can also customise the colour of the bar.
%You still need to enter your current completed titles, and You can easily do this with a command that accepts Variable Arguments. Still, that also helped you review your notes.
%It also displays a partial table of contents under every progress bar and only shows the subdirectories of the current part. 
% \section{Commands}
% \begin{macro}{\usepackage}
% \cmd{\usepackage\oarg{options}\{unifront\}} \\
% \oarg{options}: You can choose from 'red', 'blue', 'yellow', 'help'.
% \end{macro}
%
% \begin{macro}{\currentchapter}
% \cmd{\currentchapter\marg{chapter name}\marg{chapter name}...\marg{chapter name}...}\\
% You can put many \marg{chapter name} as you want after \cmd{\currentchapter} .\\
% \marg{chapter name}: The name of completed chapter.
% \end{macro}
%
% \begin{macro}{\currentsections}
% \cmd{\currentsections\marg{section name}\marg{section name}...\marg{section name}...}\\
% You can put many \marg{section name} as you want after \cmd{\currentsections}.\\
% \marg{section name}: The name of completed section.
% \end{macro}
%
% \begin{macro}{\currentsubsections}
% \cmd{\currentsubsections\marg{subsection name}...\marg{subsection name}...}\\
% You can put many \marg{subsection name} as you want after \cmd{\currentsubsections}.\\
% \marg{subsection name}: The name of completed subsection.
% \end{macro}
%
% \begin{macro}{\newtitle}
% \cmd{\newtitle\marg{title name}\marg{number of chapters}\{\cs{currentchapter}\marg{n}\marg{n}...\}}\\
% It generate the front page for your document.\\
% \marg{title name}: The name of your document.\\
% \marg{number of chapters}: The total number of the chapters you are planing to do.\\
% \{\cs{currentchapter}\marg{n}...\}: Put the name of each completed chapter in to \marg{n} in order. 
% \end{macro}
%
% \begin{macro}{\newchapter}
% \cmd{\newchapter\marg{chapter name}\marg{number of sections}\{\cs{currentsection}\marg{n}\marg{n}...\}}\\
% It generate the front page for the current chapter.\\
% \marg{chapter name}: The name of the chapter.\\
% \marg{number of sections}: The total number of the sections you are planing to do.\\
% \{\cs{currentsection}\marg{n}...\}: Put the name of each completed section in to \marg{n} in order. 
% \end{macro}
%
% \begin{macro}{\newsection}
% \cmd{\newsection\marg{section name}\marg{number of subsections}\{\cs{currentsubsection}\marg{n}\marg{n}...\}}\\
% It generate the front page for the current section.\\
% \marg{section name}: The name of the section.\\
% \marg{number of subsections}: The total number of the subsections you are planing to do.\\
% \{\cs{currentsection}\marg{n}...\}: Put the name of each completed subsection in to \marg{n} in order. 
% \end{macro}
%
% \begin{macro}{\newsubsection}
% \cmd{\newsubsection\marg{subsection name}}\\
% It give you a subsection \\
% \marg{subsection name}: The name of the subsection.
% \end{macro}
%
% \section{Example}
% The basic structure should look like the following:\footnote{Go to \url{https://github.com/weifanw/unifront} for more information}
\begin{verbatim}
%% Preset
\documentclass[a4paper, twoside]{report}

\usepackage[blue]{unifront}
\usepackage{epsfig}
\usepackage{hyperref}
\graphicspath{{images/}}

%% Document information
\author{Weifan W.}
\date{04/05/2023}

%% Document start from here
\begin{document}

\newtitle{This is a title name}{5}{\currentchapter{chapter1name}{chapter2name}}

%% Chapter1 start from here
\newchapter{chapter1name}{12}{\currentsections{section1name}{section2name}{section3name}{section4name}}

%% section1 start from here
\newsection{section1name}{5}{\currentsubsections{subsection1name}{subsection2name}}
This is a sentence.

\newsubsection{subsection1name}
This is a sentence.

\newsubsection{subsection2name}
This is a sentence.

%% section2 start from here
\newsection{section2name}{5}{\currentsubsections{subsection3name}}

\newsubsection{subsection3name}
This is a sentence.

%% section3 start from here
\newsection{section3name}{4}{\currentsubsections{subsection4name}{subsection5name}{subsection6name}{subsection7name}}
This is a sentence.

\newsubsection{subsection4name}
This is a sentence.

\newsubsection{subsection5name}
This is a sentence.

\newsubsection{subsection6name}
This is a sentence.

%% section4 start from here
\newsection{section4name}{5}{\currentsubsections{subsection7name}}

\newsubsection{subsection7name}
This is a sentence.

%% Chapter2 start from here
\newchapter{chapter2name}{9}{\currentsections{section5name}{section6name}}

%% section5 start from here
\newsection{section5name}{5}{\currentsubsections{subsection8name}{subsection9name}}

\newsubsection{subsection8name}
This is a sentence.

\newsubsection{subsection9name}
This is a sentence.

%% section6 start from here
\newsection{section6name}{5}{\currentsubsections{subsection10name}{subsection11name}}

\newsubsection{subsection10name}
This is a sentence.

\newsubsection{subsection11name}
This is a sentence.

\end{document}
\end{verbatim}
% \Finale
%
\endinput